\documentclass{homework-thesis}
\facultyname{XXX学院}
\classnumber{M123456789}
\studentname{张三}
\professional{计算机科学与技术}
\researchdirection{NLP与知识图谱}
\covertitlefirstline{第一行题目}
\covertitlesecondline{第二行题目}
\begin{document}
    \makecover
    \section{作业背景}
    \section{文献综述}
    \section{技术方案}
    \section{实验结果及分析}
    \begin{thebibliography}{99} %代表最多可以插入多少参考文献
        \bibitem{ref1} 周义棋, 田向亮, 钟茂华. 基于微博网络爬虫的巴黎圣母院大火舆情分析[J].武汉理工大学学报. 2019, 41(5): 461-466.
        \bibitem{ref2} Jing Wang, Yunchun Guo. Scrapy-based crawling and user-behavior characteristics analysis on taobao [C]. International conference on cyber-enabled distributed computing and knowledge discovery, 2012, 44-52.
        \bibitem{ref3} Sunshin Lee, Mohamed M. G. Farag, Edward A. Fox. Focused crawler for events[J]. International journal on digital libraries. 2018,19(1): 3-19.
    \end{thebibliography}
\end{document}
