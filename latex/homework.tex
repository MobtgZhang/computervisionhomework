\documentclass{homework-thesis}
\facultyname{电子电气工程学院}
\classnumber{M123456789}
\studentname{学生姓名}
\professional{专业名字}
\researchdirection{\textcolor{red}{如果未确定,就填待定}}
\covertitlefirstline{这是第一行题目}
\covertitlesecondline{这是第二行题目}
\begin{document}
    \makecover
    \section{作业背景}
    {
        \bfseries\textcolor{red}{
            内容文字为小四,宋体,1.5倍行距。注意句首缩进和两端对齐。此外,图表都应有合适的标注和解释。可以引用和借鉴别人的工作,但需要给出来源,禁止抄袭。以下为计分标准:
            \begin{enumerate}
                \item 格式符合要求,有进行代码编程实验,对结果有分析,体现一定的工作量(正文6页以上)。(80)
                \item Something interesting:在1的基础上+扎实的工作 or 严谨的证明 or 让人感兴趣的应用 or 其他与众不同的地方。(80以上)
                \item 存在抄袭,或者其他明显欺骗行为。(60以下)
                \item 最终解释权由本人掌握,可以和我讨论,但不要和我杠。
            \end{enumerate}
        }
    }
    \section{文献综述} 
	\subsection{二级标题示例}
    正文内容
    \section{技术方案}
    \section{实验结果及分析}
    \begin{thebibliography}{99} %代表最多可以插入多少参考文献
        \bibitem{ref1} 周义棋, 田向亮, 钟茂华. 基于微博网络爬虫的巴黎圣母院大火舆情分析[J].武汉理工大学学报. 2019, 41(5): 461-466.
        \bibitem{ref2} Jing Wang, Yunchun Guo. Scrapy-based crawling and user-behavior characteristics analysis on taobao [C]. International conference on cyber-enabled distributed computing and knowledge discovery, 2012, 44-52.
        \bibitem{ref3} Sunshin Lee, Mohamed M. G. Farag, Edward A. Fox. Focused crawler for events[J]. International journal on digital libraries. 2018,19(1): 3-19.
    \end{thebibliography}
\end{document}
